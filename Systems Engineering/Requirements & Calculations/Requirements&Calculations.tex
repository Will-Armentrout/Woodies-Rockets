\documentclass[12pt, letter paper]{article}

\usepackage[a4paper, left=0.9in, right=0.9in, top=1in, bottom=1in]{geometry}
\usepackage{graphics}
\usepackage[colorlinks=true, linkcolor=blue, urlcolor=blue]{hyperref}
\usepackage{gensymb}

\title{Requirements and Calculations of a Liquid Rocket Engine}
\author{Will Armentrout}
\date{September 15, 2024}

\begin{document}
	\maketitle

	\tableofcontents
	
	\section{Nomenclature and Terms}
		$ N2O $ - Nitrous Oxide \\
		$ IPA $ - Isopropyl Alcohol \\
		$I_{sp}$ - Specific Impulse\\
		$ O/F $ - Oxidizer/Fuel Ratio
		
	\section{System Requirements/Specifications}
	\begin{center}
		\begin{tabular}{|c |c |c |}
			\hline
			\textbf{Requirement} & \textbf{Value} & \textbf{From?} \\ \hline
			Thrust & 900 N & Rocket Weight \\ \hline
			Specific Impulse & 226 s & NASA CEA \\ \hline
			Tank Pressure & 5.127 MPa & Property (Vapor Pressure: 20 C) \\ \hline
			O/F Mole Ratio & 9 & Ideal Reaction \\ \hline
			O/F Mass Ratio & 6.6 & Calculated (O/F Mole Ratio) \\ \hline 
			Weight Flow Rate & 3.6 N/s & Calculated (Thrust and Isp)\\ \hline
			Mass Flow Rate & 0.4059 kg/s & Calculated (Weight Flow Rate) \\ \hline
			Ox Mass Flow Rate & 0.3525 kg/s & Calculated (Mass Flow Rate) \\ \hline
			Fuel Mass Flow Rate & 0.0534 kg/s & Calculated (Mass Flow Rate) \\ \hline
			Chamber Pressure & 3 MPa & Assumed (Tank Pressure) \\ \hline
			Chamber Temperature & 2966 K & NASA CEA \\ \hline
			Chamber Gas Density & 3.3349 kg/m3 & NASA CEA \\ \hline
			Chamber Gamma & 1.1459 & NASA CEA \\ \hline
			Chamber Mach Number & 0.00 & Assumption \\ \hline
			Chamber SoS & 1015 m/s & NASA CEA \\ \hline
			Throat Pressure & 1.7248 MPa & NASA CEA \\ \hline
			Throat Temperature & 2790 K & NASA CEA \\ \hline
			Throat Gas Density & 2.0577 kg/m3 & NASA CEA \\ \hline
			Throat Gamma & 1.1471 & NASA CEA \\ \hline
			Throat Mach Number & 1.00 & Definition \\ \hline
			Throat SoS & 981 m/s & NASA CEA \\ \hline
			Exit Pressure & 0.1 MPa & Estimated (Altitude) \\ \hline
			Exit Temperature & 1907 K & NASA CEA \\ \hline
			Exit Gas Density & 0.17948 kg/m3 & NASA CEA \\ \hline
			Exit Gamma & 1.2044 & NASA CEA \\ \hline
			Exit Mach Number & 2.708 & NASA CEA \\ \hline
			Exit SoS & 819 m/s & NASA CEA \\ \hline
			Area Ratio & 5.0683 & NASA CEA \\ \hline
			C* & 1487 m/s & NASA CEA \\ \hline
			Specific Gas Constant & 32.4638 g/mol & NASA CEA \\ \hline
			Throat Area & $ 65.591 mm^2 $ & Calculated (Isentropic Equations) \\ \hline
			Throat Diameter & 9.1385 mm & Calculated \\ \hline
			Exit Area & $ 332.4343 mm^2 $ & Calculated (Area Ratio) \\ \hline
			Exit Diameter & 20.5734 mm & Calculated \\ \hline
			Nozzle Exit Angle & 15\degree & Determined (Hobby Rocket Engines)\\ \hline
			Throat Entry Angle & 30\degree & Determined (Youngblood) \\ \hline
			Characteristic Length & 1 m & Determined (Youngblood) \\ \hline
			Chamber Diameter & 30 mm & Determined \\ \hline
			Chamber Length & 84.36 mm & Calculated (Hobby Rocket Engines) \\ \hline
			Chamber Wall Thickness & 5 mm & Calculated \& Determined \\ \hline
			
			
		\end{tabular}
	\end{center}

	\section{Derivations, Formulas \& Calculation Process}
		
		\subsection{Deriving Oxidizer/Fuel Ratio}
			Ideal Reaction: 
			$ C_3H_8O + 9N_2O \rightarrow 3CO_2 + 4H_2O + 9N_2 $ 
			\begin{center}
			\textbf{\textit{In reality the reaction is more complicated. the heat of reaction causes many radicals and exotic species that can be calculated but are ignored for simplicity.}} 
			\end{center}
			\noindent Mole O/F Ratio of 9 (9 Moles of N2O for 1 mole of C3H8O). \\
			$ M_{N2O} = 2*14 + 16 = 44 $, 
			$ M_{C3H8O} = 3*12 + 8*1 + 16 =60 $ \\ \\
			$O/F_{Mass} = \frac{9*44}{60} = 6.6 $ (6.6 kg of N2O for 1 kg of C3H8O)
		
		\subsection{NASA CEA}
			Once these variables have been determined the NASA CEA (Chemical Equalibrium with Applications) or other equalibrium solver can be used to get many other critical parameters.
			\subsubsection{Procedure for running the NASA CEA}
				This is by no means the only way to do it. This is also intended as an early design exercise where other chamber pressures and O/F ratios should be analyzed. 
				If you just want a single analysis point don't use the interval tabs and just use the number entry on the right side of the tabs for pressure and O/F ratio.
				\begin{enumerate}
					\item Look up NASA CEA on any browser. URL: \href{https://cearun.grc.nasa.gov/}{\underline{https://cearun.grc.nasa.gov/}}
					\item On the right hand side in the chemical equalibrium problem types check "Rocket". Then hit submit.
					\item On the "Pressure" tab (should be autoselected) enter values for the chamber pressure. 
						\begin{itemize}
							\item This will be based on the tank pressure (5.127 MPa) and the losses in the plumbing to the chamber. 
							\item Since MPa, isn't a selectable option, multiply the MPa number by 10 to get the pressure in bar.
							\item Assume 3 MPa as an approximate starting point. Since 3 MPa is our goal pressure, input 2 MPa (20 bar) as a Low Value and 5 MPa (50 bar) as a high value. 
							\item Make the interval smaller, maybe 0.5 MPa (5 bar) to see a range of pressure values. 
							\item Accept Input and Continue.
						\end{itemize}
					\item On the "Fuel(s)" tab check the "Use Periodic Table (mixtures)" since IPA isn't on the quick select list. Accept Fuel Selection and Continue.
					\item On the "Periodic Table" check Hydrogen, Carbon and Oxygen (as these are the component elements in IPA). Accept Element Selections and Continue.
					\item On the "Select your Fuel(s)" tab check C3H8O, 2propanol (This is the chemical formula and name for Isopropyl Alcohol). Accept Selected Reactant(s) and Continue. 
						There should be a confirmation form, Fuel Mix form and Component Properties form. Accept inputs for all and continue to the next form.
					\item On the "Select your Oxidizer(s)" tab check N2O. Accept Oxidizer Selection and Continue to the next form.
					\item On the "Enter proportions of Oxidizer/Fuel" on the top selection area select Specify with Oxidizer/Fuel Wt. Ratio. 
						\begin{itemize}
							\item We calculated the stoichometric O/F weight ratio earlier as being 6.6. 
							\item If 6.6 is the target value enter a low value of 4 and a high value of 8 and use an interval of 0.2.
						\end{itemize}
					\item On the "Define Exit Conditions" tab define Chamber/Exit Pressure Ratios on the left hand side. 
						\begin{itemize}
							\item In rocket engines the most efficient engine is one that is perfectly expanded (meaning the pressure at the exit of the nozzle is at ambient pressures).
							\item A rocket engine will fly in a multitude of different pressures as it travels upward (pressure decreases with altitude),
								but for now we will analyze with a exit pressure of 1 bar (or about 110m in altitude) as a baseline.
							\item In order to get an exit pressure of 1 bar we take our chamber pressure range (2-5 MPa in steps of 0.5 MPa or 20-50 bar in steps of 5 bar) and divide it by the desired exit pressure.
							\item So for this it's easy. The ratios are just the chamber pressure values. You will have to input each ratio manually (20, 25, 30 ... to 50) on the left column.
							\item Accept Input and Continue to Next Form.
						\end{itemize}
					\item On the "Enter Your Final Choices Before Running CEA" the autogenerated selections should be good (Short, Mass-Fractions, Equilibrium). 
						You can look at your entered ranges for chamber pressure, O/F ratio and chamber/exit pressure ratios before you run the CEA.
						Submit Input and Perform CEA Analysis.
					\item The CEA should run in less than a couple of seconds. The file displayed in the embedded window is the output file. You can download the input and output files in the top right.		
				\end{enumerate}
			\subsubsection{Procedure for analyzing the NASA CEA output}
				The output file of the NASA contains many valuable pieces of information that will be useful in designing a rocket engine later. 
				A navigation tip: the CEA output file is sorted from smallest to largest O/F ratio first and within each O/F ratio it is sorted by chamber pressure. 
				\begin{enumerate}
					\item Scroll through the output file to find the O/F Ratio of 6.6 and the Pin of 3 MPa (435.1 psi). For some reason the CEA output only displays the chamber pressures in psi no matter the input unit.
					\item Once you have found the section where Pin = 435.1 psi and O/F = 6.6 find the exit condition where the P, bar equals 1.000 or Pinf/P = 30. 
						\begin{itemize}
							\item The chamber, throat and exit pressure column where P, BAR = 1.000 (or another exit pressure) are the only columns of interest per section.
						\end{itemize}
					\item Find and record the variables of interest.
						In our case this would be pressure (P), temperature (T), density (RHO), mach number, specific heat ratio (GAMMA), speed of sound (SON VEL), area ratio (Ae/At),
						c star, specific impulse (Isp) and the mass fractions. Here are the values:
						\begin{itemize}
							\item Pressure at the throat: 17.248 Bar
							\item Temperature in the combustion chamber: 2966 K, throat: 2790 K, exit: 1907 K
							\item Density of the combustion gases in the combustion chamber: 3.3349 kg/m3, throat 2.0577 kg/m3, exit 0.17948 kg/m3
							\item Mach Number at exit: 2.708
							\item Specific Heat Ratio at the combustion chamber: 1.1459, throat: 1.1471, exit: 1.2044
							\item Speed of Sound at the combustion chamber: 1015 m/s, throat: 981 m/s, exit: 819 m/s
							\item Area Ratio: 5.0683
							\item C star: 1487 m/s
							\item Specific Impulse: 2218 m/s. In order to convert this to the familiar specific impulse with units of seconds divide by the gravitational constant g = 9.81. $ I_{sp} = \frac{2218}{9.81} = 226 s $
							\item The mass fractions are listed for the main combustion gas products.
							\begin{itemize}
								\item CO: Combustion Chamber - 0.05069, Throat -  0.03951, Exit - 0.00258
								\item CO2: Combustion Chamber - 0.20943, Throat - 0.22700, Exit - 0.28503
								\item H: Combustion Chamber - 0.00010, Throat -  0.00007, Exit - 0.00000
								\item HO2: Combustion Chamber - 0.00003, Throat - 0.00002, Exit - 0.00000
								\item H2: Combustion Chamber - 0.00083, Throat - 0.00065, Exit - 0.00006
								\item H2O: Combustion Chamber - 0.14328, Throat - 0.14690, Exit - 0.15703
								\item NO: Combustion Chamber - 0.01266, Throat - 0.00909, Exit - 0.00047
								\item NO2: Combustion Chamber - 0.00002, Throat - 0.00001, Exit - 0.00000
								\item N2: Combustion Chamber - 0.54681, Throat - 0.54849, Exit - 0.55252
								\item O: Combustion Chamber - 0.00151, Throat - 0.00093, Exit - 0.00001
								\item OH: Combustion Chamber - 0.01160, Throat - 0.00850, Exit - 0.00041
								\item O2: Combustion Chamber - 0.02302, Throat -  0.01885, Exit - 0.00190
							\end{itemize}
						\end{itemize}
				\end{enumerate}
			
		\subsection{Deriving the Flow rates}
			Specific Impulse Equation:  $I_{sp} = \frac{T}{\dot{m} g_0} $ 
			$\rightarrow$
			Weight Flow Version: $ \dot{W} = \dot{m} g_0= \frac{T}{I_{sp}}$ \\
			Solve: $ \dot{W} = \frac{900 N}{226 s} = 3.982 N/s $ \\ \\
			Convert to mass flow: $ \dot{m} = \frac{\dot{W}}{g_0} \rightarrow \dot{m} = \frac{3.9823}{9.81} = 0.4059 kg/s $ \\ \\
			Separate into oxidizer and fuel mass flow rates: \\
			Oxidizer - Nitrous Oxide: $ \dot{m}_{N2O} = \dot{m} * \frac{6.6}{7.6} = 0.3525 kg/s $ \\
			Fuel - Isopropyl Alcohol: $ \dot{m}_{IPA} = \dot{m}*\frac{1}{7.6} = 0.0534 kg/s $ \\
	
	\section{Design Equations}
		\subsection{Nozzle Design Equations}
			\subsubsection{Solving for the specific gas constant}
				The Univeral Gas Constant $ \bar{R} $ is 8.314 J/molK. \\
				The specific gas constant is $ R = \frac{\bar{R}}{M} $ where M is the molecular weight of the mixture of product gases.
				Need to solve for the molecular weight of all of the component gases and their moles in mass fraction.
				\begin{itemize}
					\item CO: $ M = 12 +16 = 28 $ $x_{CO} = 28 \cdot 0.03951 = 1.1063 $ moles
					\item CO2: $ M = 12 + 16  + 16 = 44 $ $ x_{CO2} = 44 \cdot 0.2270 = 9.988 $ moles
					\item H: $ M = 1 $ $ x_{H} = 1 \cdot 0.00007 = 0.00007 $ moles
					\item HO2: $ M = 1 + 16 + 16 = 33 $ $ x_{HO2} = 33 \cdot 0.00002 = 0.00066 $ moles
					\item H2: $ M = 1 + 1 = 2 $ $ x_{H2} = 2 \cdot 0.00065 = 0.0013 $ moles
					\item H2O: $ M = 1 + 1 + 16 = 18 $ $ x_{H2O} = 18 \cdot 0.1469 = 2.6442 $ moles
					\item NO: $ M = 14 + 16 = 30 $ $ x_{NO} = 30 \cdot 0.00909 = 0.2727 $ moles
					\item NO2: $ M = 14 + 16 + 16 = 46 $ $ x_{NO2} = 46 \cdot 0.00001 = 0.00046 $ moles
					\item N2: $ M = 14 + 14 = 28 $ $ x_{N2} = 28 \cdot 0.54849 = 15.3577 $ moles
					\item O: $ M = 16 $ $ x_{O} = 16 \cdot 0.00093 = 0.01488 $ moles
					\item OH: $ M =16 + 1 = 17 $ $ x_{OH} = 17 \cdot 0.0085 = 0.1445 $ moles
					\item O2: $ M =16 + 16  = 32 $ $ x_{O2} = 32 \cdot 0.01885 = 0.6032 $ moles
					\item Total: $ x_{total} = 1.1063 + 9.988 + 0.00007 + 0.00066 + 0.0013 + 2.6442 + 0.2727 + 0.00046 + 15.3577 + 0.01488 + 0.1445 +0.6032 = 30.1346 $ moles
				\end{itemize}
				Molecular weight of mixture is $M = 28 \cdot \frac{1.1063}{30.1346} + 44 \cdot \frac{9.988}{30.1346} + 1 \cdot \frac{0.00007}{30.1346} + 33 \cdot \frac{0.00066}{30.1346} + 2 \cdot \frac{0.0013}{30.1346} + 18 \cdot \frac{2.6442}{30.1346} + 30 \cdot \frac{0.2727}{30.1346} + 46 \cdot \frac{0.00046}{30.1346} + 28 \cdot \frac{15.3577}{30.1346} + 16 \cdot \frac{0.01488}{30.1346} + 17 \cdot \frac{0.1445}{30.1346} + 32 \cdot \frac{0.6032}{30.1346} = 32.4645 g/mol$ \\ \\
				Solve for specific gas constant $ R = \frac{8.314}{32.4645} = 0.2561 J/gK = 256.1 J/kgK $ 

			\subsubsection{Throat Area}
				Mass Flow Rate at the throat: $ \dot{m} = A_t \frac{P_o}{\sqrt{T_o}}\sqrt{\frac{\gamma}{R}} (\frac{\gamma + 1}{2})^{\frac{\gamma + 1}{2(1-\gamma)}}  $ \\
				Area of Throat: $ A_t = \frac{\dot{m}\sqrt{T_o}}{P_o}\sqrt{\frac{R}{\gamma}} (\frac{\gamma + 1}{2})^{\frac{\gamma + 1}{2(\gamma -1)}}$ \\ \\
				Values used in the Equation: \\
				Mass Flow Rate $ \dot{m} = 0.4059 kg/s $ \\
				Stagnation Temperature $ T_o = T_{chamber} = 2966 K $ \\
				Stagnation Pressure $ P_o = P_{chamber} = 3 MPa $ \\
				Gamma (at the throat) $ \gamma_{throat} = 1.1471 $ \\
				Specific Gas Constant (at the throat) $ R_{throat} = 256.1 J/kgK $ \\ \\
				Plug in the numbers: $ A_t = \frac{0.4059 \cdot \sqrt{2966}}{3 \cdot 10^6}\sqrt{\frac{256.1}{1.1471}} (\frac{1.1471 + 1}{2})^{\frac{1.1471 + 1}{2(1.1471 -1)}} = 6.5591 \cdot 10^{-5} m^2$ \\ \\
				Solve for Diameter $ A_t = \pi \frac{{D_t}^2}{4}  \rightarrow D_t = \sqrt{\frac{4A_t}{\pi}} = \sqrt{\frac{4 \cdot (6.5591 \cdot 10^{-5})}{\pi}} = 0.0091385 m = 9.1385 mm$
			
			\subsubsection{Exit Area}
				Use the throat area and determined $ A_e/A_t $ to get the exit area. $ \frac{A_e}{A_t} = 5.0683 $ \\
				$ A_e = 5.0683 \cdot 6.5591 \cdot 10^{-5} = 3.3243 \cdot 10^{-4} m^2$ \\
				Diameter: $ D_e = \sqrt{\frac{4 A_e}{\pi}} = \sqrt{\frac{4 \cdot(3.3243 \cdot 10^{-4})}{\pi}} = 0.02057 m = 20.5734 mm $ \\
			
			\subsubsection{Nozzle Angles}
				In normal commercial rocket engines the nozzle angle is optimized so that when the flow leaves it is as 1D as possible, to achieve max thrust. 
				The profile that maximizes thrust for a certain throat area and length is called an "Rao optimum contour". This is is named after G.V.R Rao and is complicated to manufacture. 
				Because of its complexity we will just use a 15{\degree} conical nozzle profile, common for hobby and small area ratio engines. It has 98.3\% of the performance of an ideal nozzle, so close enough for me.

		\subsection{Combustion Chamber Equations}
			
			\subsubsection{Chamber Length}
				For the combustion chamber Hobby Rocket Engines references a combustion chamber with a cross sectional area of at least 3 times the throat area. 
				For this reason we choose a chamber diameter of 30 mm, much larger than the required size, subject to change based on the injector. \\ \\
				The chamber volume required for complete combustion is defined by the characteristic chamber length: $ L^* = V_c/A_t \rightarrow L^* A_t = V_c $ where $ V_c = A_c L_c +$ convergent volume  \\
				Youngblood references a characterisitc length of 1m being successful in nitrous, ethanol engines so that value will be used as a minimum. \\ \\
				For small combustion chambers the convergent volume is 1/10th the cylindrical portion so $V_c = 1.1A_cL_c $ \\
				Combine Equations $ L^* A_t = 1.1A_cL_c \rightarrow L_c = \frac{L^* A_t}{1.1A_c} = \frac{(1) (6.5591 \cdot 10^{-5})}{1.1(\pi \frac{0.03^2}{4})} = 0.08436 m$ or $ 84.36 mm $
			
			\subsubsection{Chamber Thickness}
				This system will definitely not be thin walled but it is a good starting point for minimums. \\
				The axial or longnitudinal stress in a thin walled pressure vessel is given by the equation: 
				$ \sigma_a = \frac{Pr}{2t} $ Where P is the pressure, r is radius and t is the thickness. \\ \\
				Rrearrange for thickness: $ t = \frac{Pr}{2\sigma_a} $ In our case pressure is the chamber pressure (3 MPa), radius is 15 mm and sigma is material dependent. 
				Since steel is cheap we will use that, heat transfer be damned (hopefully we don't run it for long enough). Steel rod stock on McMaster has a tensile yield stress of 60,000 psi or 413.685 MPa. \\
				Plug this in:  $ t = \frac{3 \cdot 15}{2 \cdot 413.685} = 0.054389 mm $ 
				Remember this number is at ambient temperatures and the combustion chamber gas were calculated to be at 2966 K (Above the melting point of steel). \\ \\
				Recalculating with an operating temperature of 500{\degree}C or 773.15 K steel has about 40\% of it's yield strength at 21C. Using this: $ t = \frac{3 \cdot 15}{2 \cdot(0.4) (413.685) \cdot} = 0.13597 mm $ \\
				Not really much of a worry, but let's use something thick enough to handle the heat loads as well, call the wall thickness 5mm.
			

			
		

\end{document}