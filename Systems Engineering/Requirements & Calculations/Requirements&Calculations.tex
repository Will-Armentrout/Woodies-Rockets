\documentclass[12pt, letter paper]{article}

\usepackage{geometry}
\usepackage{graphics}

\title{Requirements and Calculations of the Rocket Engine}
\author{Will Armentrout}
\date{September 15, 2024}

\begin{document}
	\maketitle
	
	\begin{center}
		\textbf{System Requirements/Specifications}\\
		\begin{tabular}{|c |c |c |}
			\hline
			\textbf{Requirement} & \textbf{Value} & \textbf{From?} \\ \hline
			Thrust & 900 N & Rocket Weight \\ \hline
			Specific Impulse & 250 s & Previous Engines using IPA and N2O \\ \hline
			Tank Pressure & 5127 kPa & Material Characteristic (Vapor Pressure at 20 C) \\ \hline
			Weight Flow Rate & 3.6 N/s & Calculated \\ \hline
			
			
		\end{tabular}
	\end{center}
	\noindent\rule{\linewidth}{0.4pt}
	\begin{center}
		\textbf{Deriving the weight flow rate} \\
	\end{center}
	Specific Impulse Equation:  $I_{sp} = \frac{T}{\dot{m} g_0} $ 
	$\rightarrow$
	Weight Flow Version: $ \dot{W} = \dot{m} g_0= \frac{T}{I_{sp}}$ \\
	Solve: $ \dot{W} = \frac{900 N}{250 s} = 3.6 N/s $
	
	\noindent\rule{\linewidth}{0.4pt}
	\begin{center}
		\textbf{Deriving Oxidizer/Fuel Ratio} \\
	\end{center}
	Ideal Reaction: 
	$ C_3H_8O + 9N_2O \rightarrow 3CO_2 + 4H_2O + 9N_2 $ 
	\begin{center}
	\textbf{\textit{In reality the reaction is more complicated. the heat of reaction causes many radicals and exotic species that can be calculated but are ignored for simplicity.}} 
	\end{center}
	\noindent O/F Ratio of 9 (9 Moles of N2O for 1 mole of C3H8O). \\
	$ M_{N2O} = 2*14 + 16 = 44 $ \\
	$ M_{C3H8O} = 3*12 + 8*1 + 16 =60 $ \\	
	

\end{document}